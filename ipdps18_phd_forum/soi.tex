\subsection{Short bio sketch}
I am a third year PhD candidate in the Electrical and Computer Engineering (ECE)
department at Rutgers University and a member of the RADICAL group, lead by 
Dr. Shantenu Jha. Prior to my PhD program, I obtained my Masters in ECE working 
with the RADICAL group at Rutgers University and a Bachelors in Electronics and 
Communication Engineering from SSN College of Engineering, Anna University, 
Chennai, India.

\subsection{Current research interests}
I was introduced to the concept of abstractions and ensemble-based applications
(EBAs) in molecular dynamics during my research, towards my Masters, with the 
RADICAL group. My research consisted of a survey of existing frameworks that 
abstract resource and task management complexities from domain scientists
interested in executing EBAs on HPCs. This work culminated in the development of
Ensemble Toolkit~\cite{entk} (EnTK v0.1) and ExTASY~\cite{extasy}. My research 
investigated abstractions not just for execution of EBAs but also composition
of EBAs. This work lead to the identification of three separate layers of 
concern: (1) specification of tasks and resource requirements, (2) task-graph 
traversal and (3) resource selection and acquisition, and task execution 
management. I am interested in the concepts of component-based software 
engineering and studying their potential in enabling composability and reuse in 
similar frameworks in the HPC community~\cite{review_bb_2016}.

\subsection{Plan for future research/career}
Many life science applications are reaching the scientific limits of predefined
EBAs and require adapting the application based on intermediate results towards
relevant ranges of results. I am interested in characterizing the functional 
requirements of adaptive applications, i.e. applications with modifications to 
the task-graph during runtime. EnTK can be enhanced with such capabilities
based on the characterization.

Many applications that operate at full scale of data and parameter space often
require a large set of resources for concurrent execution of its tasks (and thus
smaller time to completion (TTC)), such that they cannot be accommodated on a 
single HPC system. In the IPDPS technical paper, we discussed the seismic 
inversion application that, at full scale, consists of 4096 simulations each 
requiring 384 GPUs. This requirement cannot be supported by a single HPC system 
currently and these simulations are executed sequentially leading to larger TTC. 
I am interested in investigating the requirements and capabilities of a framework 
that enables execution of such applications over multiple distributed computing 
infrastructures (DCI) such as HPC systems, grids and clouds. This opens up 
interesting questions in the context of scheduling tasks and data placement 
across different DCIs.

Post my PhD, I am interested in research and development opportunities in 
parallel and distributed computing, and systems engineering in the industry.


\subsection{Objectives for participating in IPDPS Student Program}

% \begin{itemize}
% \item How do you expect the interaction with the IPDPS community will further 
% your research/career goals?
% \end{itemize}

Comments for reviewers were very helpful.