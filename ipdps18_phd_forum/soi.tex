\subsection{Short bio sketch}
I am a fourth year PhD candidate in the Electrical and Computer Engineering (ECE)
department at Rutgers University and a member of the RADICAL group, lead by 
Dr. Shantenu Jha. Prior to my PhD program, I obtained my master`s degree in ECE 
working with the RADICAL group and a bachelor`s degree in Electronics and 
Communication Engineering from SSN College of Engineering, Anna University, 
Chennai, India.

% \subsection{Current research interests}
% \jhanote{What are the motivating problems that frame your research?}
% I was introduced to the concepts of abstractions for resource management 
% \jhanote{what kind of abstractions?} and ensemble-based applications (EBAs) during
% my research towards my Masters. My research consisted of a survey of existing
% frameworks that abstract resource and task management complexities from domain
% scientists interested in executing EBAs on HPCs. This work culminated in the
% development of Ensemble Toolkit~\cite{entk} (EnTK v0.1) and
% ExTASY~\cite{extasy}. In my PhD, I was introduced to applications from other
% science domains, such as seismology and climate science. My research extended
% to investigation of abstractions not just for execution of EBAs on HPCs but
% also for composition of EBAs. This work lead to the identification of three
% separate layers of concern: (1) specification of tasks and resource
% requirements, (2) resource selection and acquisition, and (3) and task
% execution management. I am interested in the concepts of component-based
% software engineering and studying their potential in enabling composability
% and reuse in similar frameworks in the HPC community~\cite{review_bb_2016}.

\subsection{Current research interests} During my master's studies, I 
studied ensemble-based applications (EBAs) in molecular dynamics and 
software abstractions for their execution on high performance computing 
(HPC) systems. In my research, I reviewed the existing frameworks that 
support execution of EBAs, focusing on how domain scientists can be 
isolated from resource and task management complexities. My research 
contributed to the development of the Ensemble Toolkit~\cite{entk} and 
ExTASY~\cite{extasy}, two software systems supporting the execution of 
EBAs on HPC systems.

In my PhD, I extended my research to support the execution of workflows from
diverse scientific domains, including molecular dynamics, seismology and
climate science. I developed abstractions to support both composition and
execution of EBAs on HPCs, identifying three main concerns: (1)
specification of tasks and resource requirements; (2) resource selection and
acquisition; and (3) task execution management. I am also interested in the
concepts of component-based software engineering, studying how they can be
used to enable composability and reuse of software frameworks
for the HPC community~\cite{review_bb_2016}.

\subsection{Plan for future research/career}
\jhanote{this sentence does not make sense ..}\vbnote{fixed}
% Many life science applications are reaching the scientific limits of predefined
% EBAs and require adapting the application based on intermediate results towards
% interesting regions of the set of possible results. I am interested in 
% characterizing the functional requirements of adaptive applications, i.e. 
% applications with modifications to the task-graph during runtime and 
% incorporating these capabilities in EnTK.

Several applications in the field of molecular dynamics benefit from steering
their sampling towards interesting molecular structures. These applications
are called adaptive because they require the capability to modify their task
graph at runtime, based on intermediate results. I am studying the requirements 
and overheads of composing and executing adaptive applications, investigating 
the similarities and differences among types of adaptivity. Ultimately, my 
research will contribute to derive a unified model of adaptive ensemble-based 
applications.

Many scientific applications require increasing the concurrency of task
execution to reduce time to completion. The resource requirements of such
applications cannot be accommodated on a single HPC system. For example, in
the paper accepted at IPDPS 2018, we discuss the seismic inversion use case
that, at full concurrency, would require ~96x capacity of ORNL Titan.

\jhanote{you've introduced scale as a the dominant challenge. then you
go to talking about smaller machines that have a different set of
challenges.} \vbnote{I am proposing using multiple machines concurrently when
one single machine is not sufficient} A solution to this problem is to enable 
the execution of such applications across multiple distributed computing 
infrastructures (DCI) such as HPC systems, grids and clouds. I am studying 
existing frameworks that enable such capabilities. EnTK will be incorporated 
with the capability to derive execution strategies based on the analysis of 
the workload and resources. This effort will target optimal task scheduling 
and data placement across different DCIs.
% Such capabilities require dedicated abstractions 
% to isolate the user from the complexity of using different DCIs. 


\jhanote{post my phd is not grammatically correct}
\vbnote{fixed, thanks}
After my PhD, I am interested in research and development opportunities in
the field of parallel and distributed computing, and systems engineering in
the industry.

\subsection{Objectives for participating in IPDPS Student Program}

Participation in the IPDPS PhD forum will provide me with a platform to share
my current research work and future research ideas with students and experts
in the parallel and distributed computing field. This will provide valuable
feedback that will be helpful to refine my research ideas, towards the
completion of my thesis. Further, sharing my research at such a venue will
enable outreach to students and communities that may benefit from the
theoretical results of my research and the software tools I have developed.

The forum will also provide me with an opportunity to engage with students
working on other interesting topics in the field. In this context, I foresee
opportunities to be introduced to other open source projects that overlap
with my interests and to which I may also be able to contribute.

I will gain a lot of valuable knowledge and experience by being part of the PhD 
forum. I am especially looking forward to talks from academic and industry 
experts on the latest research that are yet to be seen by the public. I will 
greatly benefit from the mentoring sessions and career planning provided by 
industry experts and will be able to build valuable networking in both industry 
and academia.


% \begin{itemize}
% \item How do you expect the interaction with the IPDPS community will further 
% your research/career goals?
% \end{itemize}
