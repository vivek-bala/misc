\subsection{Short bio sketch}
I am a third year PhD candidate in the Electrical and Computer Engineering (ECE)
department at Rutgers University and a member of the RADICAL group, lead by 
Dr. Shantenu Jha. Prior to my PhD program, I obtained my master`s degree in ECE 
working with the RADICAL group at Rutgers University and a bachelor`s degree in 
Electronics and Communication Engineering from SSN College of Engineering, Anna 
University, Chennai, India.

\subsection{Current research interests}
I was introduced to the concepts of abstractions and ensemble-based applications
(EBAs) in molecular dynamics during my research towards my Masters. My research 
consisted of a survey of existing frameworks that abstract resource and task 
management complexities from domain scientists interested in executing EBAs on 
HPCs. This work culminated in the development of Ensemble Toolkit~\cite{entk} 
(EnTK v0.1) and ExTASY~\cite{extasy}. In my PhD, I was introduced to 
applications from other science domains, such as seismology and climate science.
My research extended to investigation of abstractions not just for execution of 
EBAs on HPCs but also for composition of EBAs. This work lead to the 
identification of three separate layers of concern: (1) specification of tasks 
and resource requirements, (2) resource selection and acquisition, and (3) and 
task execution management. I am interested in the concepts of component-based 
software engineering and studying their potential in enabling composability and 
reuse in similar frameworks in the HPC community~\cite{review_bb_2016}.

\subsection{Plan for future research/career}
Many life science applications are reaching the scientific limits of predefined
EBAs and require adapting the application based on intermediate results towards
interesting regions of the set of possible results. I am interested in 
characterizing the functional requirements of adaptive applications, i.e. 
applications with modifications to the task-graph during runtime and 
incorporating these capabilities in EnTK.

Many applications that operate at full scale of data and parameter space often
require a large set of resources for concurrent execution of its tasks (and thus
smaller time to execution (TTE)), such that they cannot be accommodated on a 
single HPC system. In the IPDPS technical paper, we discussed the seismic 
inversion application that, at full scale, consists of 4096 simulations each 
requiring 384 GPUs. This simulations cannot be concurrently execution on a 
single HPC system currently and, consequently, are executed sequentially leading
to larger TTE. I am interested in investigating requirements and capabilities of 
frameworks that enables execution of such applications over multiple distributed 
computing infrastructures (DCI) such as HPC systems, grids and clouds. Although
this will increase the number of resources as seen by the application, it opens 
up interesting questions in the context of scheduling tasks and data placement 
across different DCIs.

Post my PhD, I am interested in research and development opportunities in 
the field of parallel and distributed computing, and systems engineering in the 
industry.


\subsection{Objectives for participating in IPDPS Student Program}

Participation in the IPDPS PhD forum will provide me a platform to share my 
current research work and future research ideas to students and experts in
the parallel and distributed computing field. I believe I will get valuable 
feedback that will be helpful to refine my research ideas. I believe sharing my 
research at such a venue will enable outreach to students and communities that 
may benefit from my research work.

The forum will provide me with an opportunity to engage with students working
on other interesting topics in the field. I also foresee opportunities to be 
introduced to other open source projects that overlap with my interests and
I may also be able to contribute.

I believe I will gain a lot of valuable knowledge and experience by being part
of the PhD forum. I am especially looking forward to talks from academic and 
industry experts on the latest research that are yet to be seen by the public. 
I will greatly benefit from the mentoring sessions and career planning directly 
with the industry experts and will be able to build valuable contacts in both 
industry and academia.

% \begin{itemize}
% \item How do you expect the interaction with the IPDPS community will further 
% your research/career goals?
% \end{itemize}
